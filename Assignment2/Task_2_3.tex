\subsection*{Task 2.3}

\begin{center}
	\begin{tabular}{ | l | l | l | l | l |}
		\hline
		Joint & $a_{i}$ & $\alpha_{i}$ & $d_i$ & $\theta_i$ \\ \hline
		1 & 0 & $90\degree$ & $d_2$ & $\theta_1$\\ \hline
		2 & 0 & $-90\degree$ & $d_2$ & $\theta_2$\\ \hline
		3 & 0 & $0\degree$ & $d^*_3$ (depends on displacement of joint 3) & $0\degree$\\ \hline
		4 & 0 & $90\degree$ & 0 & $\theta_4$\\ \hline
		5 & 0 & $90\degree$ & 0 & $\theta_5$\\ \hline
		6 & 0 & $0\degree$ & 0 & $\theta_6$\\ \hline
	\end{tabular}
\end{center}

Version Jonas:
\begin{center}
	\begin{tabular}{ | l | l | l | l | l |}
		\hline
		Joint & $a_{i}$ & $\alpha_{i}$ & $d_i$ & $\theta_i$ \\ \hline
		1 & 0 & $0\degree$ & 0 & $\theta_1$\\ \hline
		2 & 0 & $-90\degree$ & $d_2$ & $\theta_2$\\ \hline
		3 & 0 & $90\degree$ & $d^*_3$ (depends on displacement of joint 3) & $0\degree$\\ \hline
		4 & 0 & $0\degree$ & 0 & $\theta_4$\\ \hline
		5 & 0 & $90\degree$ & $d_5$ & $\theta_5$\\ \hline
		6 & 0 & $-90\degree$ & $d_6$ & $\theta_6$\\ \hline
	\end{tabular}
\end{center}
Für Joint 1 denke ich, dass der Joint parallel zur Basis ausgerichtet ist. Also $0\degree$ und kein d. Joint 2 seh ich genauso wie bei dir. Joint 3 müsste doch nochmal um $90\degree$ rotiert sein ansonsten gleich. Joint 4 denke ich soll dann auch die z-Achse wieder parallel zur vorherigen sein, also $0\degree$ und damit auch kein d. Für Joint 5 und 6 seh ich dann die beiden Rotationen und beim d weiß ich nicht ob die Angabe im Text mit $d_2$ bedeutet, dass es kein anderes d gibt. Theoretisch würde ich denken, dass es ein $d_5$ und $d_6$ geben müsste.
\\
The matrix of a total transformation of $CS_0$ to $CS_1$ in the general case is:

\begin{align*}
^{i-1}A_i &= 
\begin{pmatrix}
cos(\theta_i) & -sin(\alpha_i) & sin(\theta_i)sin(\alpha_i) & a_icos(\theta_i) \\
sin(\theta_i) & cos(\theta_i)cos(\alpha_i) & -cos(\theta_i)sin(\alpha_i) & a_isin(\theta_i) \\
0 & sin(\alpha_i) & cos(\alpha_i) & d_i \\
0 & 0 & 0 & 1 \\
\end{pmatrix}
\end{align*}

Therefore we get the following homogeneous transformations: