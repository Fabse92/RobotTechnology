\subsection*{Task 1.3}
Three parameter values are required to describe every possible orientation in 3-dimensional space.
Euler angles represent elemental rotations by $\phi,\theta,\psi$ about the axes of a coordinate system.\\

\subsubsection*{1.3.1}
Three possible Euler angle combinations:
\begin{itemize}
	\item \textbf{(x-y'-z'')}: These three rotations can be viewed as the concatenation of yaw, pitch and roll of an aircraft.
	\item \textbf{(z-y'-z'')}: The same orientation can be achieved, while using only two rotation axes, for example by doing a roll, then a pitch and another roll in the end. The first roll can be viewed as preparation for the pitch do bring the aircraft in the correct position in space, the second roll is then necessary, because the first roll was only to prepare the pitch, not to roll in the end-orientation.
	\item \textbf{(z-x'-z'')}: Instead of doing a pitch the aircraft can also do a yaw with the same explanation as before.
\end{itemize}

\subsubsection*{1.3.2}
The number of twelve possible sequences of Euler-angles can be calculated with a combinatorial consideration:\\
The first used axes is arbitrary, the second mustn't be the previous one (we lose one DOF otherwise, we could have done a rotation around the first axes with a different angle instead) and the third mustn't be the previous one again. So we come to $3\cdot 2\cdot 2 = 12$ possible sequences.