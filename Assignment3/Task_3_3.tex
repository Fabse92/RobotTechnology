\subsection*{Task 3.3}
Following transformations are given:
\begin{align*}
^{camera}T_{object} &= 
\begin{pmatrix}
0 & -1 & 0 & 0\\
-1 & 0 & 0 & -5\\
0 & 0 & -1 & 19\\
0 & 0 & 0 & 1
\end{pmatrix},
^{camera}T_{base} = 
\begin{pmatrix}
0 & -1 & 0 & 15\\
-1 & 0 & 0 & 25\\
0 & 0 & -1 & 20\\
0 & 0 & 0 & 1
\end{pmatrix}
\end{align*}
\subsubsection*{Task 3.3.1}
To calculate the transformation $^{base}T_{object}$ we need the inverse of $^{camera}T_{base}$ because:\\
$^{base}T_{object} =\,^{camera}T_{base}^{-1} \cdot\,^{camera}T_{object} =\,^{base}T_{camera} \cdot\,^{camera}T_{object}$
The inverse of a matrix can be calculated with the Gauß-Jordan-Algorithm:
\begin{align*}
(^{camera}T_{base}|I) &= 
\left(
\begin{array}{cccc|cccc}
0 & -1 & 0 & 15 & 1 & 0 & 0 & 0\\
-1 & 0 & 0 & 25 & 0 & 1 & 0 & 0\\
0 & 0 & -1 & 20 & 0 & 0 & 1 & 0\\
0 & 0 & 0  & 1   & 0 & 0 & 0 & 1
\end{array}
\right)
\rightarrow 
\left(
\begin{array}{cccc|cccc}
-1 & 0 & 0 & 25 & 0 & 1 & 0 & 0\\
0 & -1 & 0 & 15 & 1 & 0 & 0 & 0\\
0 & 0 & -1 & 20 & 0 & 0 & 1 & 0\\
0 & 0 & 0  & 1   & 0 & 0 & 0 & 1
\end{array}
\right)\\
&\rightarrow 
\left(
\begin{array}{cccc|cccc}
1 & 0 & 0 & 0 & 0 & -1 & 0 & 25\\
0 & 1 & 0 & 0 & -1 & 0 & 0 & 15\\
0 & 0 & 1 & 0 & 0 & 0 & -1 & 20\\
0 & 0 & 0  & 1 & 0 & 0 & 0 & 1
\end{array}
\right)
=(I|^{camera}T_{base}^{-1}) 
\\
^{base}T_{camera} &=
\begin{pmatrix}
0 & -1 & 0 & 25\\
-1 & 0 & 0 & 15\\
0 & 0 & -1 & 20\\
0 & 0 & 0 & 1
\end{pmatrix}
\\
^{base}T_{object} &=
\begin{pmatrix}
0 & -1 & 0 & 25\\
-1 & 0 & 0 & 15\\
0 & 0 & -1 & 20\\
0 & 0 & 0 & 1
\end{pmatrix}
\cdot
\begin{pmatrix}
0 & -1 & 0 & 0\\
-1 & 0 & 0 & -5\\
0 & 0 & -1 & 19\\
0 & 0 & 0 & 1
\end{pmatrix}
=
\begin{pmatrix}
1 & 0 & 0 & 30\\
0 & 1 & 0 & 15\\
0 & 0 & 1 & 1\\
0 & 0 & 0 & 1
\end{pmatrix}
\end{align*}

\subsubsection*{Task 3.3.2}
In respect to the base the manipulator is rotated $-90\degree$ around the z-axis and $180\degree$ around the x-axis, which is the same rotation as for the transformation from base to camera.
The translation from base to manipulator during the grasp corresponds to the translation from $^{base}T_{object}$, because their origins coincide.
Therefore we get for $^{base}T_{tool}$:
\begin{align*}
^{base}T_{tool} &=
\begin{pmatrix}
0 & -1 & 0 & 30\\
-1 & 0 & 0 & 15\\
0 & 0 & -1 & 1\\
0 & 0 & 0 & 1
\end{pmatrix}
=
\begin{pmatrix}
n_x & s_x & a_x & p_x\\
n_y & s_y & a_y & p_y\\
n_z & s_z & a_z & p_z\\
0 & 0 & 0 & 1
\end{pmatrix}
\end{align*}

\subsubsection*{Task 3.3.3}
The equations to calculate the joint angles can be retrieved from the given material about PUMA roboter arms:

\begin{align*}
\theta_1 &=atan2\left[\frac{-ARM\cdot p_y\sqrt{p_x^2+p_y^2-d_2^2}-p_xd_2}{-ARM\cdot p_x\sqrt{p_x^2+p_y^2-d_2^2}+p_yd_2}\right]\\
\theta_2 &=atan2\left[\frac{sin\alpha\cos\beta+(ARM\cdot ELBOW)cos\alpha sin\beta}{cos\alpha\cos\beta-(ARM\cdot ELBOW)sin\alpha sin\beta}\right]\\
\text{mit } sin\alpha &= -\frac{p_z}{\sqrt{p_x^2+p_y^2+p_z^2-d_2^2}}, cos\alpha = -\frac{ARM\cdot \sqrt{p_x^2+p_y^2-d_2^2}}{\sqrt{p_x^2+p_y^2+p_z^2-d_2^2}}\\
\text{und } cos\beta &= -\frac{p_x^2+p_y^2+p_z^2+a_2^2-d_2^2-(d_4^2+a_3^2)}{2a_2\sqrt{p_x^2+p_y^2+p_z^2-d_2^2}}, sin\beta = \sqrt{1-cos^2\beta}\\
\theta_3 &=atan2\left[\frac{sin\phi cos\beta-cos\phi sin\beta}{cos\phi cos\beta+sin\phi sin\beta}\right]\\
\text{mit } cos\phi &= \frac{a_2^2+(d_4^2+a_3^2)-(p_x^2+p_y^2+p_z^2-d_2^2)}{2a_2\sqrt{d_4^2+a_3^2}}, sin\phi = ARM\cdot ELBOW\sqrt{1-cos^2\phi}\\
\text{und } sin\beta &= \frac{d_4}{\sqrt{d_4^2+a_3^2}}, cos\beta = \frac{|a_3|}{\sqrt{d_4^2+a_3^2}}
\end{align*}

With $\theta_1$, $\theta_2$ and $\theta_3$ we can calculate $A_0^1$, $A_1^2$ and $A_2^3$ and then the last three angles:

\begin{align*}
\theta_4 &=atan2\left[\frac{WRIST\cdot sign(\Omega)\cdot(C_1a_y-S_1a_x)}{WRIST\cdot sign(\Omega)\cdot(C_1C_{23}a_x+S_1C_{23}a_y-S_{23}a_z)}\right]\\
\theta_5 &=atan2\left[\frac{(C_1C_{23}C_4-S_1S_4)a_x+(S_1C_{23}C_4+C_1S_4)a_y-C_4S_{23}a_z}{C_1S_{23}a_x+S_1S_{23}a_y+C_{23}a_z}\right]\\
\theta_6 &=atan2\left[\frac{(-S_1C_4-C_1C_{23}S_4)n_x+(C_1C_4-S_1C_{23}S_4)n_y+(S_4S_{23})n_z}{(-S_1C_4-C_1C_{23}S_4)s_z+(C_1C_4-S_1C_{23}S_4)s_y+(S_4S_{23})s_z}\right]
\end{align*}

To actually calculate the joint angles, we need the DH-parameter values, the $T_6=\,^{base}T_{tool}$ matrix and we need to set WRIST, ARM and ELBOW.
Using the DH-parameters for the PUMA 560 from lecture 2, we have for the non-zero parameters:
\begin{align*}
a_2 &= 0.4318m, a_3=-0.02032m, d_2=0.14909m, d_4=0.43307m
\end{align*}
From the $^{base}T_{tool}$ matrix we retrieve:
\begin{align*}
n_y = -1, s_x = -1, a_z = -1, p_x = 30, p_y = 15, p_z = 1
\end{align*}
We then choose:
\begin{align*}
WRIST = 1, ARM = 1, ELBOW = 1
\end{align*}
The first three angles are therefore calculated as:
\begin{align*}
\theta_1 &=atan2\left[\frac{-15\sqrt{30^2+15^2-0.14909^2}-30\cdot 0.14909}{-30\sqrt{30^2+15^2-0.14909^2}+15\cdot 0.14909}\right] = atan2\left[\frac{-507.5830}{-1003.9843}\right] = -153.2\degree\\
sin\alpha &= -\frac{1}{\sqrt{30^2+15^2+1^2-0.14909^2}} =-0.0298, cos\alpha = -\frac{\sqrt{30^2+15^2-0.14909^2}}{\sqrt{30^2+15^2+1^2-0.14909^2}} = -1\\
cos\beta &= -\frac{30^2+15^2+1^2+0.4318^2-0.14909^2-(0.43307^2+(-0.02032)^2)}{2\cdot 0.4318\sqrt{30^2+15^2+1^2-0.14909^2}} = 38.8554\\
sin\beta &= \sqrt{1-cos^2\beta} = \\
\theta_2 &=atan2\left[\frac{sin\alpha\cos\beta+(ARM\cdot ELBOW)cos\alpha sin\beta}{cos\alpha\cos\beta-(ARM\cdot ELBOW)sin\alpha sin\beta}\right]\\
\end{align*}