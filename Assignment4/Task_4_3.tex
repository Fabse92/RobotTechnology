\subsection*{Task 4.3}

Figure 3 on the exercise sheet directly shows two singular configurations. First, when the arm is fully extended and the axes of joint two and three perfectly align. From the fully extended case we can infer that it is a workspace boundary singularity. The same workspace boundary singularity takes effect when the joint three is rotated in the exact opposite direction, which means the arm is folded onto itself. Second the axes of the fourth and sixth joint are aligned in the joint configuration in figure 3. This can also happen when the arm is not fully extended or folded and thus is a workspace-internal singularity. The third possible singularity exists between joint one and six. The sixth joint needs to be rotated upwards by 90$\degree$ and can then rotate itself in such a way that its axes align with the axes of joint one. Again this is a workspace-internal singularity.