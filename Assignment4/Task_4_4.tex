\subsection*{Task 4.4}

To derive the homogeneous transformation $Rot_{\textbf{k},\theta}$ we consider it's effect on a vector $\textbf{v}=(v_x,v_y,v_z)^T$.
\textbf{v} can be separated in an orthogonal $\textbf{v}_o$ and a parallel part $\textbf{v}_p$ in respect to $\textbf{k}=(k_x,k_y,k_z)$ so $\textbf{v} = \textbf{v}_o+\textbf{v}_p$.
\textbf{k} lays in the origin of our coordinate system therefore $Rot_{\textbf{k},\theta}(\textbf{v}) = \textbf{v}_p+Rot_{\textbf{k},\theta}(\textbf{v}_o)$, the rotation on the parallel part of $\textbf{v}$ has no effect.
The orthogonal part of $\textbf{v}$ is then rotated about $\theta$ around $\textbf{k}$ in the the plane spanned by $\textbf{v}_o$ and $\textbf{k}\times\textbf{v}_o$.
Which translates to: $Rot_{\textbf{k},\theta}(\textbf{v}_o) = cos\theta\cdot\textbf{v}_o+sin\theta\cdot\textbf{k}\times\textbf{v}_o= cos\theta\cdot\textbf{v}_o+sin\theta\cdot\textbf{k}\times\textbf{v}$ (here we assume \textbf{k} to be normalized). $\textbf{v}_p$ can be written as $\textbf{v}_p = (\textbf{v}\cdot\textbf{k})\textbf{k}$.

The effect of the rotation around \textbf{k} is then: 

\begin{align*}
Rot_{\textbf{k},\theta}(\textbf{v}) &= (\textbf{v}\cdot\textbf{k})\textbf{k}+cos\theta\cdot\textbf{v}_o+sin\theta\cdot\textbf{k}\times\textbf{v}\\
&= (\textbf{v}\cdot\textbf{k})\textbf{k}+cos\theta\cdot(\textbf{v}-(\textbf{v}\cdot\textbf{k})\textbf{k})+sin\theta\cdot\textbf{k}\times\textbf{v}\\
&= (\textbf{v}\cdot\textbf{k})\textbf{k}+cos\theta\cdot\textbf{v}-cos\theta\cdot(\textbf{v}\cdot\textbf{k})\textbf{k}+sin\theta\cdot\textbf{k}\times\textbf{v}\\
&= (1-cos\theta)(\textbf{v}\cdot\textbf{k})\textbf{k}+cos\theta\cdot\textbf{v}+sin\theta\cdot\textbf{k}\times\textbf{v}
\end{align*}

To calculate the corresponding matrix to $Rot_{\textbf{k},\theta}$, we use the following identities:

\begin{align*}
\textbf{k}\times\textbf{v} &= 
	\begin{pmatrix}
	0 & -k_z & k_y\\
	k_z & 0 & -k_x\\
	-k_y & k_x & 0
	\end{pmatrix}\cdot \textbf{v} \\
(\textbf{v}\cdot\textbf{k})\textbf{k} &= 
\begin{pmatrix}
k_x^2 & k_xk_y & k_xk_z \\
k_xk_y & k_y^2 & k_yk_z \\
k_xk_z & k_yk_z & k_z^2
\end{pmatrix}\cdot \textbf{v}
\end{align*}

The effect of the rotation matrix can then be calculated as:

\begin{align*}
Rot_{\textbf{k},\theta}(\textbf{v}) &= (1-cos\theta)(\textbf{v}\cdot\textbf{k})\textbf{k}+cos\theta\cdot\textbf{v}+sin\theta\cdot\textbf{k}\times\textbf{v}\\
&= \left((1-cos\theta) 
\begin{pmatrix}
k_x^2 & k_xk_y & k_xk_z \\
k_xk_y & k_y^2 & k_yk_z \\
k_xk_z & k_yk_z & k_z^2
\end{pmatrix}+cos\theta\cdot
\begin{pmatrix}
1 & 0 & 0 \\
0 & 1 & 0 \\
0 & 0 & 1
\end{pmatrix}+sin\theta\cdot
\begin{pmatrix}
0 & -k_z & k_y\\
k_z & 0 & -k_x\\
-k_y & k_x & 0
\end{pmatrix}\right)\cdot\textbf{v}
\end{align*}

The rotation matrix itself can therefore be extracted as:

\begin{align*}
Rot_{\textbf{k},\theta} &= 
(1-cos\theta) 
\begin{pmatrix}
k_x^2 & k_xk_y & k_xk_z \\
k_xk_y & k_y^2 & k_yk_z \\
k_xk_z & k_yk_z & k_z^2
\end{pmatrix}+cos\theta\cdot
\begin{pmatrix}
1 & 0 & 0 \\
0 & 1 & 0 \\
0 & 0 & 1
\end{pmatrix}+sin\theta\cdot
\begin{pmatrix}
0 & -k_z & k_y\\
k_z & 0 & -k_x\\
-k_y & k_x & 0
\end{pmatrix}\\
&= 
\begin{pmatrix}
k_xk_xV\theta+C\theta & k_yk_xV\theta-k_zS\theta & k_zk_xV\theta+k_yS\theta \\
k_xk_yV\theta+k_zS\theta & k_yk_yV\theta+C\theta & k_zk_yV\theta-k_xS\theta \\
k_xk_zV\theta-k_yS\theta & k_yk_zV\theta+k_xS\theta & k_zk_zV\theta+C\theta
\end{pmatrix}
\end{align*}

Where $C\theta = cos\theta, S\theta = sin\theta$\\
and $V\theta=1-cos\theta$












