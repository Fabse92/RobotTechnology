\subsection*{Task 3.3}
Following transformations are given:
\begin{align*}
^{camera}T_{object} &= 
\begin{pmatrix}
0 & -1 & 0 & 0\\
-1 & 0 & 0 & -5\\
0 & 0 & -1 & 19\\
0 & 0 & 0 & 1
\end{pmatrix},
^{camera}T_{base} = 
\begin{pmatrix}
0 & -1 & 0 & 15\\
-1 & 0 & 0 & 25\\
0 & 0 & -1 & 20\\
0 & 0 & 0 & 1
\end{pmatrix}
\end{align*}
\subsubsection*{Task 3.3.1}
To calculate the transformation $^{base}T_{object}$ we need the inverse of $^{camera}T_{base}$ because:\\
$^{base}T_{object} =\,^{camera}T_{base}^{-1} \cdot\,^{camera}T_{object} =\,^{base}T_{camera} \cdot\,^{camera}T_{object}$
The inverse of a matrix can be calculated with the Gauß-Jordan-Algorithm:
\begin{align*}
(^{camera}T_{base}|I) &= 
\left(
\begin{array}{cccc|cccc}
0 & -1 & 0 & 15 & 1 & 0 & 0 & 0\\
-1 & 0 & 0 & 25 & 0 & 1 & 0 & 0\\
0 & 0 & -1 & 20 & 0 & 0 & 1 & 0\\
0 & 0 & 0  & 1   & 0 & 0 & 0 & 1
\end{array}
\right)
\rightarrow 
\left(
\begin{array}{cccc|cccc}
-1 & 0 & 0 & 25 & 0 & 1 & 0 & 0\\
0 & -1 & 0 & 15 & 1 & 0 & 0 & 0\\
0 & 0 & -1 & 20 & 0 & 0 & 1 & 0\\
0 & 0 & 0  & 1   & 0 & 0 & 0 & 1
\end{array}
\right)\\
&\rightarrow 
\left(
\begin{array}{cccc|cccc}
1 & 0 & 0 & 0 & 0 & -1 & 0 & 25\\
0 & 1 & 0 & 0 & -1 & 0 & 0 & 15\\
0 & 0 & 1 & 0 & 0 & 0 & -1 & 20\\
0 & 0 & 0  & 1 & 0 & 0 & 0 & 1
\end{array}
\right)
=(I|^{camera}T_{base}^{-1}) 
\\
^{base}T_{camera} &=
\begin{pmatrix}
0 & -1 & 0 & 25\\
-1 & 0 & 0 & 15\\
0 & 0 & -1 & 20\\
0 & 0 & 0 & 1
\end{pmatrix}
\\
^{base}T_{object} &=
\begin{pmatrix}
0 & -1 & 0 & 25\\
-1 & 0 & 0 & 15\\
0 & 0 & -1 & 20\\
0 & 0 & 0 & 1
\end{pmatrix}
\cdot
\begin{pmatrix}
0 & -1 & 0 & 0\\
-1 & 0 & 0 & -5\\
0 & 0 & -1 & 19\\
0 & 0 & 0 & 1
\end{pmatrix}
=
\begin{pmatrix}
1 & 0 & 0 & 30\\
0 & 1 & 0 & 15\\
0 & 0 & 1 & 1\\
0 & 0 & 0 & 1
\end{pmatrix}
\end{align*}

\subsubsection*{Task 3.3.2}
In respect to the base the manipulator is rotated $-90\degree$ around the z-axis and $180\degree$ around the x-axis, which is the same rotation as for the transformation from base to camera.
The translation from base to manipulator during the grasp corresponds to the translation from $^{base}T_{object}$, because their origins coincide.
Therefore we get for $^{base}T_{tool}$:
\begin{align*}
^{base}T_{tool} &=
\begin{pmatrix}
0 & -1 & 0 & 30\\
-1 & 0 & 0 & 15\\
0 & 0 & -1 & 1\\
0 & 0 & 0 & 1
\end{pmatrix}
\end{align*}

\subsubsection*{Task 3.3.3}




