\subsection*{Task 2.5}
The DH-parameters for the "Adept One" SCARA manipulator are:
\begin{center}
	\begin{tabular}{ | l | l | l | l | l |}
		\hline
		Joint & $a_{i}$ & $\alpha_{i}$ & $d_i$ & $\theta_i$ \\ \hline
		1 & 425mm & $\pi$ & 877mm & $q_1$\\ \hline
		2 & 375mm & $0$ & 0 & $q_2$\\ \hline
		3 & 0 & $0$ & $d_3$ & $0$\\ \hline
		4 & 0 & $0$ & 200mm & $q_4$\\ \hline
	\end{tabular}
\end{center}
\subsubsection*{Task 2.5.1}
Ich bin mir nicht so sicher, was wir machen sollen.
\begin{center}
	\begin{tabular}{ | l | l | l | l | l |}
		\hline
		Joint & $a_{i}$ & $\alpha_{i}$ & $d_i$ & $\theta_i$ \\ \hline
		1 & $a_1$ & $\pi = 180\degree$ & 0/$d_1=877$mm & $q_1$\\ \hline
		2 & $a_2$ & $0$ & 0/$d_2=0$mm & $q_2$\\ \hline
		3 & 0 & $0$ & 0/$d_3$ & $0$\\ \hline
		4 & 0 & $0$ & 0/$d_4=200$mm & $q_4$\\ \hline
	\end{tabular}
\end{center}
As all z-axes are parallel the DH-Convention is ambiguous for $d_i$.
So we can choose $d_i = 0$ everywhere but can also choose the according values. We tried to indicate this by writing 0/$d_i=value$.
\subsubsection*{Task 2.5.2}
The matrix of a total transformation of $CS_0$ to $CS_1$ in the general case is:

\begin{align*}
^{i-1}A_i &= 
\begin{pmatrix}
cos(\theta_i) & -sin(\alpha_i) & sin(\theta_i)sin(\alpha_i) & a_icos(\theta_i) \\
sin(\theta_i) & cos(\theta_i)cos(\alpha_i) & -cos(\theta_i)sin(\alpha_i) & a_isin(\theta_i) \\
0 & sin(\alpha_i) & cos(\alpha_i) & d_i \\
0 & 0 & 0 & 1 \\
\end{pmatrix}
\end{align*}

Using the DH-parameters we obtain the following homogeneous transformations:

\begin{align*}
^0A_1 &= 
\begin{pmatrix}
cos(q_1) & 0 & 0 & cos(q_1)425mm \\
sin(q_1) & -cos(q_1) & 0 & sin(q_1)425mm \\
0 & 0 & -1 & 877mm \\
0 & 0 & 0 & 1 \\
\end{pmatrix}
\end{align*}

\begin{align*}
^1A_2 &= 
\begin{pmatrix}
cos(q_2) & 0 & 0 & cos(q_2)375mm \\
sin(q_2) & cos(q_2) & 0 & sin(q_2)375mm \\
0 & 0 & 1 & 0 \\
0 & 0 & 0 & 1 \\
\end{pmatrix}
\end{align*}

\begin{align*}
^2A_3 &= 
\begin{pmatrix}
1 & 0 & 0 & 0 \\
0 & 1 & 0 & 0 \\
0 & 0 & 1 & d_3 \\
0 & 0 & 0 & 1 \\
\end{pmatrix}
\end{align*}

\begin{align*}
^3A_4 &= 
\begin{pmatrix}
cos(q_4) & 0 & 0 & 0 \\
sin(q_4) & cos(q_4) & 0 & 0 \\
0 & 0 & 1 & 200mm \\
0 & 0 & 0 & 1 \\
\end{pmatrix}
\end{align*}

and therefore:

{\scriptsize
\begin{flalign*}
^0T_4 &= \,^0A_1\cdot \,^1A_2\cdot \,^2A_3\cdot \,^3A_4 &\\ 
 &=
\begin{pmatrix}
cos(q_1) & 0 & 0 & cos(q_1)425mm \\
sin(q_1) & -cos(q_1) & 0 & sin(q_1)425mm \\
0 & 0 & -1 & 877mm \\
0 & 0 & 0 & 1 \\
\end{pmatrix}\cdot
\begin{pmatrix}
cos(q_2) & 0 & 0 & cos(q_2)375mm \\
sin(q_2) & cos(q_2) & 0 & sin(q_2)375mm \\
0 & 0 & 1 & 0 \\
0 & 0 & 0 & 1 \\
\end{pmatrix}
\cdot \,^2A_3\cdot \,^3A_4\\
 &=
\begin{pmatrix}
cos(q_1)\cdot cos(q_2) & 0 & 0 & cos(q_1)cos(q_2)375mm+cos(q_1)425mm \\
sin(q_1)\cdot cos(q_2) -cos(q_1)\cdot sin(q_2) & -cos(q_1)\cdot cos(q_2) & 0 & sin(q_1)\cdot cos(q_2)375mm-cos(q_1) \cdot sin(q_2)375mm +sin(q_1)425mm \\
0 & 0 & -1 & 877mm \\
0 & 0 & 0 & 1 \\
\end{pmatrix}\\
&\cdot
\begin{pmatrix}
1 & 0 & 0 & 0 \\
0 & 1 & 0 & 0 \\
0 & 0 & 1 & d_3 \\
0 & 0 & 0 & 1 \\
\end{pmatrix}
\cdot \,^3A_4\\
&=
\begin{pmatrix}
cos(q_1)\cdot cos(q_2) & 0 & 0 & cos(q_1)cos(q_2)375mm+cos(q_1)425mm \\
sin(q_1)\cdot cos(q_2) -cos(q_1)\cdot sin(q_2) & -cos(q_1)\cdot cos(q_2) & 0 & sin(q_1)\cdot cos(q_2)375mm-cos(q_1) \cdot sin(q_2)375mm +sin(q_1)425mm \\
0 & 0 & -1 & 877mm-d_3 \\
0 & 0 & 0 & 1 \\
\end{pmatrix}\\
&\cdot
\begin{pmatrix}
cos(q_4) & 0 & 0 & 0 \\
sin(q_4) & cos(q_4) & 0 & 0 \\
0 & 0 & 1 & 200mm \\
0 & 0 & 0 & 1 \\
\end{pmatrix}\\
&=
\begin{pmatrix}
c(q_1)c(q_2)c(q_4) & 0 & 0 & c(q_1)c(q_2)375mm+c(q_1)425mm \\
s(q_1)c(q_2)c(q_4) -c(q_1)s(q_2)c(q_4)-c(q_1)c(q_2)s(q_4) & -c(q_1)c(q_2)c(q_4) & 0 & s(q_1)c(q_2)375mm-c(q_1)s(q_2)375mm+s(q_1)425mm \\
0 & 0 & -1 & 677mm-d_3 \\
0 & 0 & 0 & 1 \\
\end{pmatrix}\\
\end{flalign*}
}
\subsubsection*{Task 2.5.3}
We use the matrix derived in task 2.5.2 and set the values accordingly to the given vector:
$\begin{pmatrix}
\pi /4 & -\pi /3 & 120 & \pi /2
\end{pmatrix}^T $

\begin{comment}
$
{\tiny
\begin{pmatrix}
c(\pi /4)c(-\pi /3)c(\pi /2) & 0 & 0 & c(\pi /4)c(-\pi /3)375mm+c(\pi /4)425mm \\
s(\pi /4)c(-\pi /3)c(\pi /2) -c(\pi /4)s(-\pi /3)c(\pi /2)-c(\pi /4)c(-\pi /3)s(\pi /2) & -c(\pi /4)c(-\pi /3)c(\pi /2) & 0 & s(\pi /4)c(-\pi /3)375mm-c(\pi /4)s(-\pi /3)375mm+s(\pi /4)425mm \\
0 & 0 & -1 & 677mm-120mm \\
0 & 0 & 0 & 1 \\
\end{pmatrix}\\
}$
\end{comment}
Which leads to the following table:
$\begin{pmatrix}
0 & 0 & 0 & 433.103mm \\
-0.3536 & 0 & 0 & 662.743mm \\
0 & 0 & -1 & 557mm\\
0 & 0 & 0 & 1\\
\end{pmatrix}$
\begin{comment}
$=
\begin{pmatrix}
90\degree & 90\degree & 90\degree & 433.103mm \\
110.7048\degree & 90\degree & 90\degree & 662.743mm\\
90\degree & 90\degree & 180\degree & 557mm\\
0 & 0 & 0 & 1\\
\end{pmatrix}$
\end{comment}
\\
Vergess ich hier noch irgendwas? Muss man das noch mit etwas Multiplizieren?