%  DOCUMENT CLASS
\documentclass[11pt]{article}

%PACKAGES
\usepackage[utf8]{inputenc}
%\usepackage[ngerman]{babel}
\usepackage[reqno,fleqn]{amsmath}
\setlength\mathindent{10mm}
\usepackage{amssymb}
\usepackage{amsthm}
\usepackage{color}
\usepackage{delarray}
% \usepackage{fancyhdr}
\usepackage{units}
\usepackage{times, eurosym}
\usepackage{verbatim} %Für Verwendung von multiline Comments mittels \begin{comment}...\end{comment}
\usepackage{wasysym} % Für Smileys
\usepackage{gensymb} % Für \degree
\usepackage{graphicx}
\usepackage{tikz}

% FORMATIERUNG
\usepackage[paper=a4paper,left=25mm,right=25mm,top=25mm,bottom=25mm]{geometry}
\usepackage{array}
\usepackage{fancybox} %zum Einrahmen von Formeln
\setlength{\parindent}{0cm}
\setlength{\parskip}{1mm plus1mm minus1mm}

\allowdisplaybreaks[1]


% PAGESTYLE

%MATH SHORTCUTS
\newcommand{\NN}{\mathbb N}
\newcommand{\ZZ}{\mathbb Z}
\newcommand{\QQ}{\mathbb Q}
\newcommand{\RR}{\mathbb R}
\newcommand{\CC}{\mathbb C}
\newcommand{\KK}{\mathbb K}
\newcommand{\U}{\mathbb O}
\newcommand{\eqx}{\overset{!}{=}}
\newcommand{\Det}{\mathrm{Det}}
\newcommand{\Gl}{\mathrm{Gl}}
\newcommand{\diag}{\mathrm{diag}}
\newcommand{\sign}{\mathrm{sign}}
\newcommand{\rang}{\mathrm{rang}}
\newcommand{\cond}{\mathrm{cond}_{\| \cdot \|}}
\newcommand{\conda}{\mathrm{cond}_{\| \cdot \|_1}}
\newcommand{\condb}{\mathrm{cond}_{\| \cdot \|_2}}
\newcommand{\condi}{\mathrm{cond}_{\| \cdot \|_\infty}}
\newcommand{\eps}{\epsilon}

\setlength{\extrarowheight}{1ex}

\begin{document}
	
	\begin{center}
		\textbf{
			Exercises: Introduction to Robotics, SS 2016\\
			Assignment \#2\\
		}
		
		\begin{tabular}{lll}
			& \\
			by & Jonas Papmeier & Mat Nr. 6326394\\
			& Jan Fabian Schmid & Mat.Nr. 6440383\\
			\\
			\hline
		\end{tabular}
	\end{center}
	
	\subsection*{Task 2.1}

\subsubsection*{Task 2.1.1}

We've got three homogeneous transformations each with a rotation $R_i$ around the z-Axis and a translation $a_i$.

\begin{align*}
^0A_1 &= 
\begin{pmatrix}
C_1 & -S_1 & 0 & C_1a_1 \\
S_1 & C_1 & 0 & S_1a_1 \\
0 & 0 & 1 & 0 \\
0 & 0 & 0 & 1 
\end{pmatrix},
\,^1A_2 = 
\begin{pmatrix}
C_2 & -S_2 & 0 & C_2a_2 \\
S_2 & C_2 & 0 & S_2a_2 \\
0 & 0 & 1 & 0 \\
0 & 0 & 0 & 1 
\end{pmatrix} \\
^2A_3 &= 
\begin{pmatrix}
C_3 & -S_3 & 0 & C_3a_3 \\
S_3 & C_3 & 0 & S_3a_3 \\
0 & 0 & 1 & 0 \\
0 & 0 & 0 & 1 
\end{pmatrix}
=
\begin{pmatrix}
C(180\degree-\theta_1-\theta_2) & -S(180\degree-\theta_1-\theta_2) & 0 & C(180\degree-\theta_1-\theta_2)a_3 \\
S(180\degree-\theta_1-\theta_2) & C(180\degree-\theta_1-\theta_2) & 0 & S(180\degree-\theta_1-\theta_2)a_3 \\
0 & 0 & 1 & 0 \\
0 & 0 & 0 & 1 
\end{pmatrix} \\
&\text{with } S(180\degree-\alpha) = S(\alpha), C(180\degree-\alpha) = -C(\alpha)\\
& \text{and } C_{12} = C(\theta_1+\theta_2) = -C(180\degree-\theta_1-\theta_2) = -C_3,\\
& S_{12} = S(\theta_1+\theta_2) = S(180\degree-\theta_1-\theta_2) = S_3\\
&=
\begin{pmatrix}
-C_{12} & -S_{12} & 0 & -C_{12}a_3 \\
S_{12} & -C_{12} & 0 & S_{12}a_3 \\
0 & 0 & 1 & 0 \\
0 & 0 & 0 & 1 
\end{pmatrix}
=
\begin{pmatrix}
-C_1C_2+S_1S_2 & -S_1C_2-C_1S_2 & 0 & (-C_1C_2+S_1S_2)a_3 \\
S_1C_2+C_1S_2 & -C_1C_2+S_1S_2 & 0 & (S_1C_2+C_1S_2)a_3 \\
0 & 0 & 1 & 0 \\
0 & 0 & 0 & 1 
\end{pmatrix} \\
^0T_3 &= \,^0A_1\cdot\,^1A_2\cdot\,^2A_3 =
\begin{pmatrix}
C_1 & -S_1 & 0 & C_1a_1 \\
S_1 & C_1 & 0 & S_1a_1 \\
0 & 0 & 1 & 0 \\
0 & 0 & 0 & 1 
\end{pmatrix}\cdot
\begin{pmatrix}
C_2 & -S_2 & 0 & C_2a_2 \\
S_2 & C_2 & 0 & S_2a_2 \\
0 & 0 & 1 & 0 \\
0 & 0 & 0 & 1 
\end{pmatrix}\cdot
\begin{pmatrix}
-C_{12} & -S_{12} & 0 & -C_{12}a_3 \\
S_{12} & -C_{12} & 0 & S_{12}a_3 \\
0 & 0 & 1 & 0 \\
0 & 0 & 0 & 1 
\end{pmatrix}\\
&= 
\begin{pmatrix}
C_1C_2-S_1S_2 & -C_1S_2-S_1C_2 & 0 & C_1C_2a_2-S_1S_2a_2+C_1a_1 \\
S_1C_2+C_1S_2 & -S_1S_2+C_1C_2 & 0 & S_1C_2a_2+C_1S_2a_2+S_1a_1 \\
0 & 0 & 1 & 0 \\
0 & 0 & 0 & 1 
\end{pmatrix}\cdot
\begin{pmatrix}
-C_{12} & -S_{12} & 0 & -C_{12}a_3 \\
S_{12} & -C_{12} & 0 & S_{12}a_3 \\
0 & 0 & 1 & 0 \\
0 & 0 & 0 & 1 
\end{pmatrix}\\
&= 
\begin{pmatrix}
C_{12} & -S_{12} & 0 & C_{12}a_2+C_1a_1 \\
S_{12} & C_{12} & 0 & S_{12}a_2+S_1a_1 \\
0 & 0 & 1 & 0 \\
0 & 0 & 0 & 1 
\end{pmatrix}\cdot
\begin{pmatrix}
-C_{12} & -S_{12} & 0 & -C_{12}a_3 \\
S_{12} & -C_{12} & 0 & S_{12}a_3 \\
0 & 0 & 1 & 0 \\
0 & 0 & 0 & 1 
\end{pmatrix}\\
&= 
\begin{pmatrix}
-C_{12}C_{12}-S_{12}S_{12} & -C_{12}S_{12}+S_{12}C_{12} & 0 & (-C_{12}C_{12}-S_{12}S_{12})a_3+C_{12}a_2+C_1a_1 \\
-S_{12}C_{12}+C_{12}S_{12} & -S_{12}S_{12}-C_{12}C_{12} & 0 & (-S_{12}C_{12}+C_{12}S_{12})a_3+S_{12}a_2+S_1a_1 \\
0 & 0 & 1 & 0 \\
0 & 0 & 0 & 1 
\end{pmatrix}
\end{align*}
Using the following equations
\begin{align*}
-C_{12}C_{12}-S_{12}S_{12} &= C(180\degree-(\theta_1+\theta_2))C(\theta_1+\theta_2)-S(180\degree-(\theta_1+\theta_2))S(\theta_1+\theta_2)\\
&= C((180\degree-\theta_1-\theta_2)+\theta_1+\theta_2) = C(180\degree) = -1\\
-C_{12}S_{12}+S_{12}C_{12} &= C(180\degree-(\theta_1+\theta_2))S(\theta_1+\theta_2)+S(180\degree-(\theta_1+\theta_2))C(\theta_1+\theta_2)\\
&= S((180\degree-\theta_1-\theta_2)+\theta_1+\theta_2) = S(180\degree) = 0\\
\end{align*}
we get
\begin{align*}
^0T_3
&= 
\begin{pmatrix}
	-1 & 0 & 0 & -a_3+C_{12}a_2+C_1a_1 \\
	0 & -1 & 0 & S_{12}a_2+S_1a_1 \\
	0 & 0 & 1 & 0 \\
	0 & 0 & 0 & 1 
\end{pmatrix}
= 
\begin{pmatrix}
-1 & 0 & 0 & C_1a_1-C_3a_2-a_3 \\
0 & -1 & 0 & S_1a_1+S_3a_2 \\
0 & 0 & 1 & 0 \\
0 & 0 & 0 & 1 
\end{pmatrix}
\end{align*}

\subsubsection*{Task 2.1.2}
\begin{align*}
^{-1}A_0 &= 
\begin{pmatrix}
1 & 0 & 0 & 0 \\
0 & C_0 & -S_0 & 0 \\
0 & S_0 & C_0 & 0 \\
0 & 0 & 0 & 1 
\end{pmatrix},
\,^{3}A_4 =
\begin{pmatrix}
1 & 0 & 0 & 0 \\
0 & C_4 & -S_4 & 0 \\
0 & S_4 & C_4 & 0 \\
0 & 0 & 0 & 1 
\end{pmatrix}
\end{align*}
Was übersehe ich hier, Jonas?

	
	\subsection*{Task 2.2}
\begin{center}
	\begin{tabular}{ | l | l | l | l | l |}
		\hline
		Joint & $a_{i}$ & $\alpha_{i}$ & $d_i$ & $\theta_i$ \\ \hline
		1 & 0 & $90\degree$ & 0.1 & $0\degree$\\ \hline
		2 & 0.7 & $0\degree$ & undefined (e.g. 0) & $0\degree$\\ \hline
		3 & 0.6 & $0\degree$ & undefined (e.g. 0) & $0\degree$\\ \hline
	\end{tabular}
\end{center}
Ich würde für Theta entweder jeweils $\theta_1$,$\theta_2$ ... oder $q_1$ ...
eintragen, da Theta Variabel für rotational joints ist. Ich bin mir nicht sicher ob und wie man den aller ersten Joint eintragen soll. Aber wir haben ja 4-DOF also müssten wir doch auch 4 Joints erhalten???
	
	\subsection*{Task 2.3}
ToDo
	
	\subsection*{Task 2.4}
\begin{center}
	\begin{tabular}{ | l | l | l | l | l |}
		\hline
		Joint & $a_{i}$ & $\alpha_{i}$ & $d_i$ & $\theta_i$ \\ \hline
		1 & $L_1$ & $0\degree$ & 0 & $-\theta_1$\\ \hline
		2 & $L_2$ & $0\degree$ & 0 & $-\theta_2$\\ \hline
		3 & $L_3$ & $0\degree$ & 0 & $-\theta_3$\\ \hline
	\end{tabular}
\end{center}

The matrix of a total transformation of $CS_0$ to $CS_1$ in the general case is:

\begin{align*}
^{i-1}A_i &= 
\begin{pmatrix}
cos(\theta_i) & -sin(\alpha_i) & sin(\theta_i)sin(\alpha_i) & a_icos(\theta_i) \\
sin(\theta_i) & cos(\theta_i)cos(\alpha_i) & -cos(\theta_i)sin(\alpha_i) & a_isin(\theta_i) \\
0 & sin(\alpha_i) & cos(\alpha_i) & d_i \\
0 & 0 & 0 & 1 \\
\end{pmatrix}
\end{align*}

When we use the properties of sine and cosine of being antisymmetric and symmetric, we get the following homogeneous transformations:

\begin{align*}
^{i-1}A_i &= 
\begin{pmatrix}
cos(\theta_i) & sin(\theta_i) & 0 & L_icos(\theta_i) \\
-sin(\theta_i) & cos(\theta_i) & 0 & -L_isin(\theta_i) \\
0 & 0 & 1 & 0 \\
0 & 0 & 0 & 1 \\
\end{pmatrix}
\end{align*}
	
	\subsection*{Task 2.5}
The DH-parameters for the "Adept One" SCARA manipulator are:
\begin{center}
	\begin{tabular}{ | l | l | l | l | l |}
		\hline
		Joint & $a_{i}$ & $\alpha_{i}$ & $d_i$ & $\theta_i$ \\ \hline
		1 & 425mm & $\pi$ & 877mm & $q_1$\\ \hline
		2 & 375mm & $0$ & 0 & $q_2$\\ \hline
		3 & 0 & $0$ & $d_3$ & $0$\\ \hline
		4 & 0 & $0$ & 200mm & $q_4$\\ \hline
	\end{tabular}
\end{center}
\subsubsection*{Task 2.5.1}
Ich bin mir nicht so sicher, was wir machen sollen.
\begin{center}
	\begin{tabular}{ | l | l | l | l | l |}
		\hline
		Joint & $a_{i}$ & $\alpha_{i}$ & $d_i$ & $\theta_i$ \\ \hline
		1 & $a_1$ & $\pi = 180\degree$ & 0/$d_1=877$mm & $q_1$\\ \hline
		2 & $a_2$ & $0$ & 0/$d_2=0$mm & $q_2$\\ \hline
		3 & 0 & $0$ & 0/$d_3$ & $0$\\ \hline
		4 & 0 & $0$ & 0/$d_4=200$mm & $q_4$\\ \hline
	\end{tabular}
\end{center}
As all z-axes are parallel the DH-Convention is ambiguous for $d_i$.
So we can choose $d_i = 0$ everywhere but can also choose the according values. We tried to indicate this by writing 0/$d_i=value$.
\subsubsection*{Task 2.5.2}
The matrix of a total transformation of $CS_0$ to $CS_1$ in the general case is:

\begin{align*}
^{i-1}A_i &= 
\begin{pmatrix}
cos(\theta_i) & -sin(\alpha_i) & sin(\theta_i)sin(\alpha_i) & a_icos(\theta_i) \\
sin(\theta_i) & cos(\theta_i)cos(\alpha_i) & -cos(\theta_i)sin(\alpha_i) & a_isin(\theta_i) \\
0 & sin(\alpha_i) & cos(\alpha_i) & d_i \\
0 & 0 & 0 & 1 \\
\end{pmatrix}
\end{align*}

Using the DH-parameters we obtain the following homogeneous transformations:

\begin{align*}
^0A_1 &= 
\begin{pmatrix}
cos(q_1) & 0 & 0 & cos(q_1)425mm \\
sin(q_1) & -cos(q_1) & 0 & sin(q_1)425mm \\
0 & 0 & -1 & 877mm \\
0 & 0 & 0 & 1 \\
\end{pmatrix}
\end{align*}

\begin{align*}
^1A_2 &= 
\begin{pmatrix}
cos(q_2) & 0 & 0 & cos(q_2)375mm \\
sin(q_2) & cos(q_2) & 0 & sin(q_2)375mm \\
0 & 0 & 1 & 0 \\
0 & 0 & 0 & 1 \\
\end{pmatrix}
\end{align*}

\begin{align*}
^2A_3 &= 
\begin{pmatrix}
1 & 0 & 0 & 0 \\
0 & 1 & 0 & 0 \\
0 & 0 & 1 & d_3 \\
0 & 0 & 0 & 1 \\
\end{pmatrix}
\end{align*}

\begin{align*}
^3A_4 &= 
\begin{pmatrix}
cos(q_4) & 0 & 0 & 0 \\
sin(q_4) & cos(q_4) & 0 & 0 \\
0 & 0 & 1 & 200mm \\
0 & 0 & 0 & 1 \\
\end{pmatrix}
\end{align*}

and therefore:

{\scriptsize
\begin{flalign*}
^0T_4 &= \,^0A_1\cdot \,^1A_2\cdot \,^2A_3\cdot \,^3A_4 &\\ 
 &=
\begin{pmatrix}
cos(q_1) & 0 & 0 & cos(q_1)425mm \\
sin(q_1) & -cos(q_1) & 0 & sin(q_1)425mm \\
0 & 0 & -1 & 877mm \\
0 & 0 & 0 & 1 \\
\end{pmatrix}\cdot
\begin{pmatrix}
cos(q_2) & 0 & 0 & cos(q_2)375mm \\
sin(q_2) & cos(q_2) & 0 & sin(q_2)375mm \\
0 & 0 & 1 & 0 \\
0 & 0 & 0 & 1 \\
\end{pmatrix}
\cdot \,^2A_3\cdot \,^3A_4\\
 &=
\begin{pmatrix}
cos(q_1)\cdot cos(q_2) & 0 & 0 & cos(q_1)cos(q_2)375mm+cos(q_1)425mm \\
sin(q_1)\cdot cos(q_2) -cos(q_1)\cdot sin(q_2) & -cos(q_1)\cdot cos(q_2) & 0 & sin(q_1)\cdot cos(q_2)375mm-cos(q_1) \cdot sin(q_2)375mm +sin(q_1)425mm \\
0 & 0 & -1 & 877mm \\
0 & 0 & 0 & 1 \\
\end{pmatrix}\\
&\cdot
\begin{pmatrix}
1 & 0 & 0 & 0 \\
0 & 1 & 0 & 0 \\
0 & 0 & 1 & d_3 \\
0 & 0 & 0 & 1 \\
\end{pmatrix}
\cdot \,^3A_4\\
&=
\begin{pmatrix}
cos(q_1)\cdot cos(q_2) & 0 & 0 & cos(q_1)cos(q_2)375mm+cos(q_1)425mm \\
sin(q_1)\cdot cos(q_2) -cos(q_1)\cdot sin(q_2) & -cos(q_1)\cdot cos(q_2) & 0 & sin(q_1)\cdot cos(q_2)375mm-cos(q_1) \cdot sin(q_2)375mm +sin(q_1)425mm \\
0 & 0 & -1 & 877mm-d_3 \\
0 & 0 & 0 & 1 \\
\end{pmatrix}\\
&\cdot
\begin{pmatrix}
cos(q_4) & 0 & 0 & 0 \\
sin(q_4) & cos(q_4) & 0 & 0 \\
0 & 0 & 1 & 200mm \\
0 & 0 & 0 & 1 \\
\end{pmatrix}\\
&=
\begin{pmatrix}
c(q_1)c(q_2)c(q_4) & 0 & 0 & c(q_1)c(q_2)375mm+c(q_1)425mm \\
s(q_1)c(q_2)c(q_4) -c(q_1)s(q_2)c(q_4)-c(q_1)c(q_2)s(q_4) & -c(q_1)c(q_2)c(q_4) & 0 & s(q_1)c(q_2)375mm-c(q_1)s(q_2)375mm+s(q_1)425mm \\
0 & 0 & -1 & 677mm-d_3 \\
0 & 0 & 0 & 1 \\
\end{pmatrix}\\
\end{flalign*}
}
\subsubsection*{Task 2.5.3}
We use the matrix derived in task 2.5.2 and set the values accordingly to the given vector:
$\begin{pmatrix}
\pi /4 & -\pi /3 & 120 & \pi /2
\end{pmatrix}^T $

\begin{comment}
$
{\tiny
\begin{pmatrix}
c(\pi /4)c(-\pi /3)c(\pi /2) & 0 & 0 & c(\pi /4)c(-\pi /3)375mm+c(\pi /4)425mm \\
s(\pi /4)c(-\pi /3)c(\pi /2) -c(\pi /4)s(-\pi /3)c(\pi /2)-c(\pi /4)c(-\pi /3)s(\pi /2) & -c(\pi /4)c(-\pi /3)c(\pi /2) & 0 & s(\pi /4)c(-\pi /3)375mm-c(\pi /4)s(-\pi /3)375mm+s(\pi /4)425mm \\
0 & 0 & -1 & 677mm-120mm \\
0 & 0 & 0 & 1 \\
\end{pmatrix}\\
}$
\end{comment}
Which leads to the following table:
$\begin{pmatrix}
0 & 0 & 0 & 433.103mm \\
-0.3536 & 0 & 0 & 662.743mm \\
0 & 0 & -1 & 557mm\\
0 & 0 & 0 & 1\\
\end{pmatrix}$
\begin{comment}
$=
\begin{pmatrix}
90\degree & 90\degree & 90\degree & 433.103mm \\
110.7048\degree & 90\degree & 90\degree & 662.743mm\\
90\degree & 90\degree & 180\degree & 557mm\\
0 & 0 & 0 & 1\\
\end{pmatrix}$
\end{comment}
\\
Vergess ich hier noch irgendwas? Muss man das noch mit etwas Multiplizieren?
	
\end{document}
