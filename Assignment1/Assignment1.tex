%  DOCUMENT CLASS
\documentclass[11pt]{article}

%PACKAGES
\usepackage[utf8]{inputenc}
%\usepackage[ngerman]{babel}
\usepackage[reqno,fleqn]{amsmath}
\setlength\mathindent{10mm}
\usepackage{amssymb}
\usepackage{amsthm}
\usepackage{color}
\usepackage{delarray}
% \usepackage{fancyhdr}
\usepackage{units}
\usepackage{times, eurosym}
\usepackage{verbatim} %Für Verwendung von multiline Comments mittels \begin{comment}...\end{comment}
\usepackage{wasysym} % Für Smileys
\usepackage{gensymb} % Für \degree


% FORMATIERUNG
\usepackage[paper=a4paper,left=25mm,right=25mm,top=25mm,bottom=25mm]{geometry}
\usepackage{array}
\usepackage{fancybox} %zum Einrahmen von Formeln
\setlength{\parindent}{0cm}
\setlength{\parskip}{1mm plus1mm minus1mm}

\allowdisplaybreaks[1]


% PAGESTYLE

%MATH SHORTCUTS
\newcommand{\NN}{\mathbb N}
\newcommand{\ZZ}{\mathbb Z}
\newcommand{\QQ}{\mathbb Q}
\newcommand{\RR}{\mathbb R}
\newcommand{\CC}{\mathbb C}
\newcommand{\KK}{\mathbb K}
\newcommand{\U}{\mathbb O}
\newcommand{\eqx}{\overset{!}{=}}
\newcommand{\Det}{\mathrm{Det}}
\newcommand{\Gl}{\mathrm{Gl}}
\newcommand{\diag}{\mathrm{diag}}
\newcommand{\sign}{\mathrm{sign}}
\newcommand{\rang}{\mathrm{rang}}
\newcommand{\cond}{\mathrm{cond}_{\| \cdot \|}}
\newcommand{\conda}{\mathrm{cond}_{\| \cdot \|_1}}
\newcommand{\condb}{\mathrm{cond}_{\| \cdot \|_2}}
\newcommand{\condi}{\mathrm{cond}_{\| \cdot \|_\infty}}
\newcommand{\eps}{\epsilon}

\setlength{\extrarowheight}{1ex}

\begin{document}
	
	\begin{center}
		\textbf{
			Exercises: Introduction to Robotics, SS 2016\\
			Assignment \#1\\
		}
		
		\begin{tabular}{lll}
			& \\
			by & Jonas Papmeier & Mat Nr. ???\\
			& Jan Fabian Schmid & Mat.Nr. 6440383\\
			\\
			\hline
		\end{tabular}
	\end{center}
	
	\subsection*{Task 1.1}

The positions of the vertices A to E in respect to the origin M are:
\begin{align*}
A = 
\begin{pmatrix}
- 6\,cm \\
- 6\,cm \\
0\,cm
\end{pmatrix}, 
B =
\begin{pmatrix}
  6\,cm \\
- 6\,cm \\
0\,cm
\end{pmatrix},
C =
\begin{pmatrix}
6\,cm \\
6\,cm \\
0\,cm
\end{pmatrix},
D =
\begin{pmatrix}
- 6\,cm \\
6\,cm \\
0\,cm
\end{pmatrix},
E =
\begin{pmatrix}
0\,cm \\
0\,cm \\
40\,cm
\end{pmatrix}
\end{align*}

The three rotations that will be applied are:
\begin{align*}
R_1 = 
\begin{pmatrix}
cos\phi & -sin\phi & 0 & 0 \\
sin\phi & cos\phi & 0 & 0 \\
0 & 0 & 1 & 0 \\
0 & 0 & 0 & 1 
\end{pmatrix}, 
R_2 = 
\begin{pmatrix}
1 & 0 & 0 & 0 \\
0 & cos\psi & -sin\psi & 0 \\
0 & sin\psi & cos\psi & 0 \\
0 & 0 & 0 & 1 
\end{pmatrix}, 
R_3 = 
\begin{pmatrix}
cos\theta & 0 & sin\theta & 0 \\
0 & 1 & 0 & 0 \\
-sin\theta & 0 & cos\theta & 0 \\
0 & 0 & 0 & 1 
\end{pmatrix}
\end{align*}

With $\phi = 30\degree, \psi = 45\degree \text{ and } \theta = -30\degree$

\subsubsection*{1.1.1}

The location of a vertex V after the sequence of rotations $R_1, R_2 \text{ and } R_3$ using the \textbf{intrinsic} rotation axes $M_w, M_u\ \text{ and } M_v$ is calculated as:

\begin{align*}
V' &= R_1\cdot R_2\cdot R_3\cdot V = R_{123} \cdot V \\
R_{123} &= 
\begin{pmatrix}
cos\phi & -sin\phi & 0 & 0 \\
sin\phi & cos\phi & 0 & 0 \\
0 & 0 & 1 & 0 \\
0 & 0 & 0 & 1 
\end{pmatrix}
\cdot
\begin{pmatrix}
1 & 0 & 0 & 0 \\
0 & cos\psi & -sin\psi & 0 \\
0 & sin\psi & cos\psi & 0 \\
0 & 0 & 0 & 1 
\end{pmatrix} 
\cdot
\begin{pmatrix}
cos\theta & 0 & sin\theta & 0 \\
0 & 1 & 0 & 0 \\
-sin\theta & 0 & cos\theta & 0 \\
0 & 0 & 0 & 1 
\end{pmatrix}\\
 &=
\begin{pmatrix}
cos\phi & -sin\phi\cdot cos\psi & sin\phi\cdot sin\psi & 0 \\
sin\phi & cos\phi\cdot cos\psi & -cos\phi\cdot sin\psi & 0 \\
0 & sin\psi & cos\psi  & 0 \\
0 & 0 & 0 & 1  
\end{pmatrix}
\cdot
\begin{pmatrix}
cos\theta & 0 & sin\theta & 0 \\
0 & 1 & 0 & 0 \\
-sin\theta & 0 & cos\theta & 0 \\
0 & 0 & 0 & 1 
\end{pmatrix}\\
 &=
 \begin{pmatrix}
 cos\phi\cdot cos\theta-sin\phi\cdot sin\psi \cdot sin\theta & -sin\phi\cdot cos\psi & cos\phi \cdot sin\theta + sin\phi\cdot sin\psi \cdot cos\theta & 0 \\
 sin\phi \cdot cos\theta+cos\phi\cdot sin\psi\cdot sin\theta & cos\phi\cdot cos\psi &  sin\phi \cdot sin\theta -cos\phi\cdot sin\psi \cdot cos\theta & 0 \\
 -cos\psi \cdot sin\theta & sin\psi & cos\psi \cdot cos\theta  & 0 \\
 0 & 0 & 0 & 1  
 \end{pmatrix}\\
 &=
 \begin{pmatrix}
 \sqrt{0.75}\cdot \sqrt{0.75}-0.5\cdot \sqrt{0.5} \cdot (-0.5) & -0.5\cdot \sqrt{0.5} & \sqrt{0.75} \cdot (-0.5) + 0.5\cdot \sqrt{0.5} \cdot \sqrt{0.75} & 0 \\
 0.5 \cdot \sqrt{0.75}+\sqrt{0.75}\cdot \sqrt{0.5}\cdot (-0.5) & \sqrt{0.75}\cdot \sqrt{0.5} &  0.5 \cdot (-0.5) -\sqrt{0.75}\cdot \sqrt{0.5} \cdot \sqrt{0.75} & 0 \\
 -\sqrt{0.5} \cdot (-0.5) & \sqrt{0.5} & \sqrt{0.5} \cdot \sqrt{0.75}  & 0 \\
 0 & 0 & 0 & 1  
 \end{pmatrix}\\
 &=
 \begin{pmatrix}
0.9268 & -\sqrt{0.125} & -0.1268 & 0 \\
0.1268 & \sqrt{0.375} &  -0.7803 & 0 \\
\sqrt{0.125} & \sqrt{0.5} & \sqrt{0.375}  & 0 \\
 0 & 0 & 0 & 1  
 \end{pmatrix}
\end{align*}

Therefore the locations of the vertices after the rotation $R_{123}$ are:

\begin{align*}
A' &= R_{123} \cdot A \\
   &= 
\begin{pmatrix}
0.9268 & -\sqrt{0.125} & -0.1268 & 0 \\
0.1268 & \sqrt{0.375} &  -0.7803 & 0 \\
\sqrt{0.125} & \sqrt{0.5} & \sqrt{0.375}  & 0 \\
0 & 0 & 0 & 1  
\end{pmatrix}
\cdot 
\begin{pmatrix}
- 6\,cm \\
- 6\,cm \\
0\,cm \\
0
\end{pmatrix}
=
\begin{pmatrix}
- 3.439\,cm \\
- 4.435\,cm \\
-6.364\,cm \\
0
\end{pmatrix}\\
B' &= R_{123} \cdot B \\
&= 
\begin{pmatrix}
0.9268 & -\sqrt{0.125} & -0.1268 & 0 \\
0.1268 & \sqrt{0.375} &  -0.7803 & 0 \\
\sqrt{0.125} & \sqrt{0.5} & \sqrt{0.375}  & 0 \\
0 & 0 & 0 & 1  
\end{pmatrix}
\cdot 
\begin{pmatrix}
6\,cm \\
- 6\,cm \\
0\,cm \\
0
\end{pmatrix}
=
\begin{pmatrix}
7.682\,cm \\
- 2.913\,cm \\
-2.121\,cm \\
0
\end{pmatrix}\\
C' &= R_{123} \cdot C \\
&= 
\begin{pmatrix}
0.9268 & -\sqrt{0.125} & -0.1268 & 0 \\
0.1268 & \sqrt{0.375} &  -0.7803 & 0 \\
\sqrt{0.125} & \sqrt{0.5} & \sqrt{0.375}  & 0 \\
0 & 0 & 0 & 1  
\end{pmatrix}
\cdot 
\begin{pmatrix}
6\,cm \\
6\,cm \\
0\,cm \\
0
\end{pmatrix}
=
\begin{pmatrix}
3.439\,cm \\
4.435\,cm \\
6.364\,cm \\
0
\end{pmatrix}\\
D' &= R_{123} \cdot D \\
&= 
\begin{pmatrix}
0.9268 & -\sqrt{0.125} & -0.1268 & 0 \\
0.1268 & \sqrt{0.375} &  -0.7803 & 0 \\
\sqrt{0.125} & \sqrt{0.5} & \sqrt{0.375}  & 0 \\
0 & 0 & 0 & 1  
\end{pmatrix}
\cdot 
\begin{pmatrix}
- 6\,cm \\
6\,cm \\
0\,cm \\
0
\end{pmatrix}
=
\begin{pmatrix}
-7.682\,cm \\
2.913\,cm \\
2.121\,cm \\
0
\end{pmatrix}\\
E' &= R_{123} \cdot E \\
&= 
\begin{pmatrix}
0.9268 & -\sqrt{0.125} & -0.1268 & 0 \\
0.1268 & \sqrt{0.375} &  -0.7803 & 0 \\
\sqrt{0.125} & \sqrt{0.5} & \sqrt{0.375}  & 0 \\
0 & 0 & 0 & 1  
\end{pmatrix}
\cdot 
\begin{pmatrix}
	0\,cm \\
	0\,cm \\
	40\,cm \\
0
\end{pmatrix}
=
\begin{pmatrix}
- 5.072\,cm \\
- 31.212\,cm \\
24.495\,cm \\
0
\end{pmatrix}\\
\end{align*}

\subsubsection*{1.1.2}

The location of a vertex V after the sequence of rotations $R_1, R_2 \text{ and } R_3$ using the \textbf{extrinsic} rotation axes $M_z, M_x \text{ and } M_y$ is calculated as:

\begin{align*}
V' &= R_3\cdot R_2\cdot R_1\cdot V = R_{321} \cdot V  \\
R_{321} &= 
\begin{pmatrix}
cos\theta & 0 & sin\theta & 0 \\
0 & 1 & 0 & 0 \\
-sin\theta & 0 & cos\theta & 0 \\
0 & 0 & 0 & 1 
\end{pmatrix}
\cdot
\begin{pmatrix}
1 & 0 & 0 & 0 \\
0 & cos\psi & -sin\psi & 0 \\
0 & sin\psi & cos\psi & 0 \\
0 & 0 & 0 & 1 
\end{pmatrix} 
\cdot
\begin{pmatrix}
cos\phi & -sin\phi & 0 & 0 \\
sin\phi & cos\phi & 0 & 0 \\
0 & 0 & 1 & 0 \\
0 & 0 & 0 & 1 
\end{pmatrix}\\
&=
\begin{pmatrix}
cos\theta & sin\theta \cdot sin\psi & sin\theta \cdot cos\psi& 0 \\
0 & cos\psi & -sin\psi & 0 \\
-sin\theta & cos\theta \cdot sin\psi & cos\theta \cdot cos\psi & 0 \\
0 & 0 & 0 & 1 
\end{pmatrix}
\cdot
\begin{pmatrix}
cos\phi & -sin\phi & 0 & 0 \\
sin\phi & cos\phi & 0 & 0 \\
0 & 0 & 1 & 0 \\
0 & 0 & 0 & 1 
\end{pmatrix}\\
&=
\begin{pmatrix}
cos\theta \cdot cos\phi + sin\theta \cdot sin\psi \cdot sin\phi  & -cos\theta \cdot sin\phi + sin\theta \cdot sin\psi \cdot cos\phi & sin\theta \cdot cos\psi& 0 \\
cos\psi \cdot sin\phi & cos\psi \cdot cos\phi & -sin\psi & 0 \\
-sin\theta \cdot cos\phi + cos\theta \cdot sin\psi \cdot sin\phi & sin\theta \cdot sin\phi + cos\theta \cdot sin\psi \cdot cos\phi & cos\theta \cdot cos\psi & 0 \\
0 & 0 & 0 & 1 
\end{pmatrix}\\
&=
\begin{pmatrix}
\sqrt{0.75} \cdot \sqrt{0.75} -0.5 \cdot \sqrt{0.5} \cdot 0.5  & -\sqrt{0.75} \cdot 0.5 -0.5 \cdot \sqrt{0.5} \cdot \sqrt{0.75} & -0.5 \cdot \sqrt{0.5}& 0 \\
\sqrt{0.5} \cdot 0.5 & \sqrt{0.5} \cdot \sqrt{0.75} & -\sqrt{0.5} & 0 \\
0.5 \cdot \sqrt{0.75} + \sqrt{0.75} \cdot \sqrt{0.5} \cdot 0.5 & -0.5 \cdot 0.5 + \sqrt{0.75} \cdot \sqrt{0.5} \cdot \sqrt{0.75} & \sqrt{0.75} \cdot \sqrt{0.5} & 0 \\
0 & 0 & 0 & 1 
\end{pmatrix}\\
&=
\begin{pmatrix}
0.5732  & -0.7391 & -\sqrt{0.125}& 0 \\
\sqrt{0.125} & \sqrt{0.375} & -\sqrt{0.5} & 0 \\
0.7391 & 0.2803 & \sqrt{0.375} & 0 \\
0 & 0 & 0 & 1 
\end{pmatrix}
\end{align*}

Therefore the locations of the vertices after the rotation $R_{321}$ are:

\begin{align*}
A' &= R_{321} \cdot A \\
&= 
\begin{pmatrix}
0.5732  & -0.7391 & -\sqrt{0.125}& 0 \\
\sqrt{0.125} & \sqrt{0.375} & -\sqrt{0.5} & 0 \\
0.7391 & 0.2803 & \sqrt{0.375} & 0 \\
0 & 0 & 0 & 1 
\end{pmatrix}
\cdot 
\begin{pmatrix}
- 6\,cm \\
- 6\,cm \\
0\,cm \\
0
\end{pmatrix}
=
\begin{pmatrix}
0.995\,cm \\
-5.796\,cm \\
-6.116\,cm \\
0
\end{pmatrix}\\
B' &= R_{321} \cdot B \\
&= 
\begin{pmatrix}
0.5732  & -0.7391 & -\sqrt{0.125}& 0 \\
\sqrt{0.125} & \sqrt{0.375} & -\sqrt{0.5} & 0 \\
0.7391 & 0.2803 & \sqrt{0.375} & 0 \\
0 & 0 & 0 & 1 
\end{pmatrix}
\cdot 
\begin{pmatrix}
6\,cm \\
- 6\,cm \\
0\,cm \\
0
\end{pmatrix}
=
\begin{pmatrix}
7.874\,cm \\
-1.553\,cm \\
2.753\,cm \\
0
\end{pmatrix}\\
C' &= R_{321} \cdot C \\
&= 
\begin{pmatrix}
0.5732  & -0.7391 & -\sqrt{0.125}& 0 \\
\sqrt{0.125} & \sqrt{0.375} & -\sqrt{0.5} & 0 \\
0.7391 & 0.2803 & \sqrt{0.375} & 0 \\
0 & 0 & 0 & 1 
\end{pmatrix}
\cdot 
\begin{pmatrix}
6\,cm \\
6\,cm \\
0\,cm \\
0
\end{pmatrix}
=
\begin{pmatrix}
-0.995\,cm \\
5.796\,cm \\
6.116\,cm \\
0
\end{pmatrix}\\
D' &= R_{321} \cdot D \\
&= 
\begin{pmatrix}
0.5732  & -0.7391 & -\sqrt{0.125}& 0 \\
\sqrt{0.125} & \sqrt{0.375} & -\sqrt{0.5} & 0 \\
0.7391 & 0.2803 & \sqrt{0.375} & 0 \\
0 & 0 & 0 & 1 
\end{pmatrix}
\cdot 
\begin{pmatrix}
- 6\,cm \\
6\,cm \\
0\,cm \\
0
\end{pmatrix}
=
\begin{pmatrix}
-7.874\,cm \\
1.553\,cm \\
-2.753\,cm \\
0
\end{pmatrix}\\
E' &= R_{321} \cdot E \\
&= 
\begin{pmatrix}
0.5732  & -0.7391 & -\sqrt{0.125}& 0 \\
\sqrt{0.125} & \sqrt{0.375} & -\sqrt{0.5} & 0 \\
0.7391 & 0.2803 & \sqrt{0.375} & 0 \\
0 & 0 & 0 & 1 
\end{pmatrix}
\cdot 
\begin{pmatrix}
0\,cm \\
0\,cm \\
40\,cm \\
0
\end{pmatrix}
=
\begin{pmatrix}
-14.142\,cm \\
-28.284\,cm \\
24.495\,cm \\
0
\end{pmatrix}\\
\end{align*}


	
	\subsection*{Task 1.2}
ToDo

	
	\subsection*{Task 1.3}
Three parameter values are required to describe every possible orientation in 3-dimensional space.
Euler angles represent elemental rotations by $\phi,\theta,\psi$ about the axes of a coordinate system.\\

\subsubsection*{1.3.1}
Three possible Euler angle combinations:
\begin{itemize}
	\item \textbf{(x-y'-z'')}: These three rotations can be viewed as the concatenation of yaw, pitch and roll of an aircraft.
	\item \textbf{(z-y'-z'')}: The same orientation can be achieved, while using only two rotation axes, for example by doing a roll, then a pitch and another roll in the end. The first roll can be viewed as preparation for the pitch do bring the aircraft in the correct position in space, the second roll is then necessary, because the first roll was only to prepare the pitch, not to roll in the end-orientation.
	\item \textbf{(z-x'-z'')}: Instead of doing a pitch the aircraft can also do a yaw with the same explanation as before.
\end{itemize}

\subsubsection*{1.3.2}
The number of twelve possible sequences of Euler-angles can be calculated with a combinatorial consideration:\\
The first used axis is arbitrary, the second mustn't be the previous one (we lose one DOF otherwise, we could have done a rotation around the first axis with a different angle instead) and the third mustn't be the previous one again. So we come to $3\cdot 2\cdot 2 = 12$ possible sequences.
	
\end{document}
