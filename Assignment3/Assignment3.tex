%  DOCUMENT CLASS
\documentclass[11pt]{article}

%PACKAGES
\usepackage[utf8]{inputenc}
%\usepackage[ngerman]{babel}
\usepackage[reqno,fleqn]{amsmath}
\setlength\mathindent{10mm}
\usepackage{amssymb}
\usepackage{amsthm}
\usepackage{color}
\usepackage{delarray}
% \usepackage{fancyhdr}
\usepackage{units}
\usepackage{times, eurosym}
\usepackage{verbatim} %Für Verwendung von multiline Comments mittels \begin{comment}...\end{comment}
\usepackage{wasysym} % Für Smileys
\usepackage{gensymb} % Für \degree
\usepackage{graphicx}
\usepackage{tikz}
\usepackage{mathtools}

% FORMATIERUNG
\usepackage[paper=a4paper,left=25mm,right=25mm,top=25mm,bottom=25mm]{geometry}
\usepackage{array}
\usepackage{fancybox} %zum Einrahmen von Formeln
\setlength{\parindent}{0cm}
\setlength{\parskip}{1mm plus1mm minus1mm}

\allowdisplaybreaks[1]


% PAGESTYLE

%MATH SHORTCUTS
\newcommand{\NN}{\mathbb N}
\newcommand{\ZZ}{\mathbb Z}
\newcommand{\QQ}{\mathbb Q}
\newcommand{\RR}{\mathbb R}
\newcommand{\CC}{\mathbb C}
\newcommand{\KK}{\mathbb K}
\newcommand{\U}{\mathbb O}
\newcommand{\eqx}{\overset{!}{=}}
\newcommand{\Det}{\mathrm{Det}}
\newcommand{\Gl}{\mathrm{Gl}}
\newcommand{\diag}{\mathrm{diag}}
\newcommand{\sign}{\mathrm{sign}}
\newcommand{\rang}{\mathrm{rang}}
\newcommand{\cond}{\mathrm{cond}_{\| \cdot \|}}
\newcommand{\conda}{\mathrm{cond}_{\| \cdot \|_1}}
\newcommand{\condb}{\mathrm{cond}_{\| \cdot \|_2}}
\newcommand{\condi}{\mathrm{cond}_{\| \cdot \|_\infty}}
\newcommand{\eps}{\epsilon}

\setlength{\extrarowheight}{1ex}

\begin{document}
	
	\begin{center}
		\textbf{
			Exercises: Introduction to Robotics, SS 2016\\
			Assignment \#3\\
		}
		
		\begin{tabular}{lll}
			& \\
			by & Jonas Papmeier & Mat Nr. 6326394\\
			& Jan Fabian Schmid & Mat.Nr. 6440383\\
			\\
			\hline
		\end{tabular}
	\end{center}
	
	\subsection*{Task 3.1}


	
	\subsection*{Task 3.2}

\subsubsection*{Task 3.2.1}
The DH-Parameters for the given manipulator are:
\begin{center}
	\begin{tabular}{ | l | l | l | l | l |}
		\hline
		Joint & $a_{i}$ & $\alpha_{i}$ & $d_i$ & $\theta_i$ \\ \hline
		1 & 0 & $-90\degree$ & $d_1$ & $\theta_1$\\ \hline
		2 & $l_a$ & $0\degree$ & 0 or $l_b$ & $\theta_2$\\ \hline
		3 & 0 & $90\degree$ & $l_c$ & $\theta_3$\\ \hline
	\end{tabular}
\end{center}

\subsubsection*{Task 3.2.2}
The matrix of a total transformation of $CS_{i-1}$ to $CS_i$ in the general case is:

\begin{align*}
^{i-1}A_i &= 
\begin{pmatrix}
cos(\theta_i) & -sin(\alpha_i)cos(\alpha_i) & sin(\theta_i)sin(\alpha_i) & a_icos(\theta_i) \\
sin(\theta_i) & cos(\theta_i)cos(\alpha_i) & -cos(\theta_i)sin(\alpha_i) & a_isin(\theta_i) \\
0 & sin(\alpha_i) & cos(\alpha_i) & d_i \\
0 & 0 & 0 & 1 \\
\end{pmatrix}
\end{align*}

\begin{align*}
^{0}A_1 &= 
\begin{pmatrix}
cos(\theta_1) & 0 & -sin(\theta_1) & 0 \\
sin(\theta_1) & 0 & cos(\theta_1) & 0 \\
0 & -1 & 0 & d_1 \\
0 & 0 & 0 & 1 \\
\end{pmatrix}
\end{align*}

\begin{align*}
^{1}A_2 &= 
\begin{pmatrix}
cos(\theta_2) & 0 & 0 & l_acos(\theta_2) \\
sin(\theta_2) & cos(\theta_2) & 0 & l_asin(\theta_2) \\
0 & 0 & 1 & l_b \\
0 & 0 & 0 & 1 \\
\end{pmatrix}
\end{align*}

\begin{align*}
^{2}A_3 &= 
\begin{pmatrix}
cos(\theta_3) & 0 & sin(\theta_3) & 0 \\
sin(\theta_3) & 0 & -cos(\theta_3) & 0 \\
0 & 1 & 0 & l_c \\
0 & 0 & 0 & 1 \\
\end{pmatrix}
\end{align*}

$T_3 = \,^{0}A_1\,^{1}A_2\,^{2}A_3$

{\scriptsize
\begin{align*}
&= 
\begin{pmatrix}
cos(\theta_1)cos(\theta_2)cos(\theta_3) & -sin(\theta_3) & cos(\theta_1)cos(\theta_2)sin(\theta_3) & -sin(\theta_1)l_c+l_acos(\theta_1)cos(\theta_2)-sin(\theta_1)l_b \\
sin(\theta_1)cos(\theta_2)cos(\theta_3) & cos(\theta_1) & sin(\theta_1)cos(\theta_2)sin(\theta_3) & cos(\theta_1)l_c+l_asin(\theta_1)cos(\theta_2)+cos(\theta_1)l_b \\
-sin(\theta_2)cos(\theta_3)-cos(\theta_2)sin(\theta_3) & 0 & -sin(\theta_2)sin(\theta_3)+cos(\theta_2)cos(\theta_3) & -l_acos(\theta_2)+d_1 \\
0 & 0 & 0 & 1 \\
\end{pmatrix}
\end{align*}
}

	
	\subsection*{Task 3.3}
Following transformations are given:
\begin{align*}
^{camera}T_{object} &= 
\begin{pmatrix}
0 & -1 & 0 & 0\\
-1 & 0 & 0 & -5\\
0 & 0 & -1 & 19\\
0 & 0 & 0 & 1
\end{pmatrix},
^{camera}T_{base} = 
\begin{pmatrix}
0 & -1 & 0 & 15\\
-1 & 0 & 0 & 25\\
0 & 0 & -1 & 20\\
0 & 0 & 0 & 1
\end{pmatrix}
\end{align*}
\subsubsection*{Task 3.3.1}
To calculate the transformation $^{base}T_{object}$ we need the inverse of $^{camera}T_{base}$ because:\\
$^{base}T_{object} =\,^{camera}T_{base}^{-1} \cdot\,^{camera}T_{object} =\,^{base}T_{camera} \cdot\,^{camera}T_{object}$
The inverse of a matrix can be calculated with the Gauß-Jordan-Algorithm:
\begin{align*}
(^{camera}T_{base}|I) &= 
\left(
\begin{array}{cccc|cccc}
0 & -1 & 0 & 15 & 1 & 0 & 0 & 0\\
-1 & 0 & 0 & 25 & 0 & 1 & 0 & 0\\
0 & 0 & -1 & 20 & 0 & 0 & 1 & 0\\
0 & 0 & 0  & 1   & 0 & 0 & 0 & 1
\end{array}
\right)
\rightarrow 
\left(
\begin{array}{cccc|cccc}
-1 & 0 & 0 & 25 & 0 & 1 & 0 & 0\\
0 & -1 & 0 & 15 & 1 & 0 & 0 & 0\\
0 & 0 & -1 & 20 & 0 & 0 & 1 & 0\\
0 & 0 & 0  & 1   & 0 & 0 & 0 & 1
\end{array}
\right)\\
&\rightarrow 
\left(
\begin{array}{cccc|cccc}
1 & 0 & 0 & 0 & 0 & -1 & 0 & 25\\
0 & 1 & 0 & 0 & -1 & 0 & 0 & 15\\
0 & 0 & 1 & 0 & 0 & 0 & -1 & 20\\
0 & 0 & 0  & 1 & 0 & 0 & 0 & 1
\end{array}
\right)
=(I|^{camera}T_{base}^{-1}) 
\\
^{base}T_{camera} &=
\begin{pmatrix}
0 & -1 & 0 & 25\\
-1 & 0 & 0 & 15\\
0 & 0 & -1 & 20\\
0 & 0 & 0 & 1
\end{pmatrix}
\\
^{base}T_{object} &=
\begin{pmatrix}
0 & -1 & 0 & 25\\
-1 & 0 & 0 & 15\\
0 & 0 & -1 & 20\\
0 & 0 & 0 & 1
\end{pmatrix}
\cdot
\begin{pmatrix}
0 & -1 & 0 & 0\\
-1 & 0 & 0 & -5\\
0 & 0 & -1 & 19\\
0 & 0 & 0 & 1
\end{pmatrix}
=
\begin{pmatrix}
1 & 0 & 0 & 30\\
0 & 1 & 0 & 15\\
0 & 0 & 1 & 1\\
0 & 0 & 0 & 1
\end{pmatrix}
\end{align*}

\subsubsection*{Task 3.3.2}
In respect to the base the manipulator is rotated $-90\degree$ around the z-axis and $180\degree$ around the x-axis, which is the same rotation as for the transformation from base to camera.
The translation from base to manipulator during the grasp corresponds to the translation from $^{base}T_{object}$, because their origins coincide.
Therefore we get for $^{base}T_{tool}$:
\begin{align*}
^{base}T_{tool} &=
\begin{pmatrix}
0 & -1 & 0 & 30\\
-1 & 0 & 0 & 15\\
0 & 0 & -1 & 1\\
0 & 0 & 0 & 1
\end{pmatrix}
\end{align*}

\subsubsection*{Task 3.3.3}





	
	\subsection*{Task 3.4}
The positioning accuracy especially depends on differences between the specified model and the actual robot. Deviations in the joint-offset and link length will propagate from the description via URDF to the derived homogeneous transformation matrices and create a discrepancy between the calculated TCP and the actual TCP. The effect on the positioning accuracy is then dependent on joint configuration. An additional factor can be inaccuracy of joint movement from the slackness of the joints. Such inaccuracy can be influenced by age of the robot and maybe even load.

The limit of the positioning accuracy will be determined by the repeatability. As the repeatability describes the distance between repeated movements targeting the same position, the positioning accuracy can in average only reach the same distance towards a target position.
	
\end{document}
