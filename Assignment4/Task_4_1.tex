\subsection*{Task 4.1}

The Jacobian is calculated as:

\begin{align*}
	J &= 
	\begin{pmatrix}
		J_v\\
		J_w
	\end{pmatrix}
	= 
	\begin{pmatrix}
		J_{v_1} & J_{v_2} & J_{v_3}\\
		J_{w_1}&J_{w_2}&J_{w_3}
	\end{pmatrix}\\
	J_{v_i} &=
	\begin{cases}
	 z_{i-1} & \text{if i is prismatic} \\
	 z_{i-1}\times (o_n^0-o_{i-1}^0) & \text{if i is revolute}
	\end{cases}\\
	J_{w_i} &=
	\begin{cases}
	0 & \text{if i is prismatic} \\
	z_{i-1} & \text{if i is revolute}
	\end{cases}
\end{align*}

To determine the $z_i$ and $o_i$, we have to find all $^0T_i$.
Therefore we calculate the DH-parameters:

\begin{center}
	\begin{tabular}{ | l | l | l | l | l |}
		\hline
		Joint & $a_{i}$ & $\alpha_{i}$ & $d_i$ & $\theta_i$ \\ \hline
		1 & $l_1$ & 0 & 0 & $\theta_1$ \\ \hline
		2 & $l_2$ & 0 & 0 & $\theta_2$ \\ \hline
		3 & $l_3$ & 0 & 0 & $\theta_3$ \\ \hline
	\end{tabular}
\end{center}

Therefore we obtain:

\begin{align*}
^0T_1 &= 
\begin{pmatrix}
C1 & -S1 & 0 & l_1 C1\\
S1 &  C1 & 0 & l_1 S1\\
0  &  0  & 1 & 0\\
0  &  0  & 0 & 1
\end{pmatrix},
\,^1T_2 = 
\begin{pmatrix}
C2 & -S2 & 0 & l_2 C2\\
S2 &  C2 & 0 & l_2 S2\\
0  &  0  & 1 & 0\\
0  &  0  & 0 & 1
\end{pmatrix},
\,^2T_3 = 
\begin{pmatrix}
C3 & -S3 & 0 & l_3 C3\\
S3 &  C3 & 0 & l_3 S3\\
0  &  0  & 1 & 0\\
0  &  0  & 0 & 1
\end{pmatrix}\\
^0T_2 &= 
\begin{pmatrix}
C1C2-S1S2 & -C1S2-S1C2 & 0 & C1l_2C2-S1l_2S2+l_1C1\\
S1C2+C1S2 & -S1S2+C1C2 & 0 & S1l_2C2+C1l_2S2+l_1S1\\
0  &  0  & 1 & 0\\
0  &  0  & 0 & 1
\end{pmatrix}
\end{align*}

{\tiny
	\begin{flalign*}
	\hspace*{-3cm}
^0T_3 &= 
\begin{pmatrix}
C1C2C3-S1S2C3-C1S2S3-S1C2S3 & -C1C2S3+S1S2S3-C1S2C3-S1C2C3 & 0 & (C1C2-S1S2)l_3 C3-(C1S2+S1C2)l_3 S3+C1l_2C2-S1l_2S2+l_1C1\\
S1C2C3+C1S2C3-S1S2S3+C1C2S3 & -S1C2S3-C1S2S3-S1S2C3+C1C2C3 & 0 & (S1C2+C1S2)l_3 C3-(S1S2-C1C2)l_3 S3+S1l_2C2+C1l_2S2+l_1S1\\
0  &  0  & 1 & 0\\
0  &  0  & 0 & 1
\end{pmatrix}
	\end{flalign*}
}

The $z_i$ and $o_i$ are then:

\begin{align*}
z_1 &= 
\begin{pmatrix}
0 \\ 0 \\ 1
\end{pmatrix},
z_2 = 
\begin{pmatrix}
0 \\ 0 \\ 1
\end{pmatrix},
z_3 = 
\begin{pmatrix}
0 \\ 0 \\ 1
\end{pmatrix}\\
o_1^0 &= 
\begin{pmatrix}
l_1C1 \\ l_1S1 \\ 0
\end{pmatrix},
o_2^0 = 
\begin{pmatrix}
C1l_2C2-S1l_2S2+l_1C1 \\ S1l_2C2+C1l_2S2+l_1S1 \\ 0
\end{pmatrix}\\
o_3^0 &= 
\begin{pmatrix}
(C1C2-S1S2)l_3 C3-(C1S2+S1C2)l_3 S3+C1l_2C2-S1l_2S2+l_1C1 \\ (S1C2+C1S2)l_3 C3-(S1S2-C1C2)l_3 S3+S1l_2C2+C1l_2S2+l_1S1 \\ 0
\end{pmatrix}
\end{align*}

As all joints are revolute, we get for the elements of the Jacobian:

\begin{align*}
J_{v_1} &= 
z_1\times (o_3^0-o_1^0)\\
 &=
\begin{pmatrix}
0 \\ 0 \\ 1
\end{pmatrix}\times\left(
\begin{pmatrix}
(C1C2-S1S2)l_3 C3-(C1S2+S1C2)l_3 S3+C1l_2C2-S1l_2S2+l_1C1 \\ (S1C2+C1S2)l_3 C3-(S1S2-C1C2)l_3 S3+S1l_2C2+C1l_2S2+l_1S1 \\ 0
\end{pmatrix}-
\begin{pmatrix}
l_1C1 \\ l_1S1 \\ 0
\end{pmatrix}
\right)\\
&=
\begin{pmatrix}
0 \\ 0 \\ 1
\end{pmatrix}\times
\begin{pmatrix}
(C1C2-S1S2)l_3 C3-(C1S2+S1C2)l_3 S3+C1l_2C2-S1l_2S2 \\ (S1C2+C1S2)l_3 C3-(S1S2-C1C2)l_3 S3+S1l_2C2+C1l_2S2 \\ 0
\end{pmatrix}
\\
&=
\begin{pmatrix}
-((S1C2+C1S2)l_3 C3-(S1S2-C1C2)l_3 S3+S1l_2C2+C1l_2S2) \\ (C1C2-S1S2)l_3 C3-(C1S2+S1C2)l_3 S3+C1l_2C2-S1l_2S2 \\ 0
\end{pmatrix}\\
J_{v_2} &= 
z_2\times (o_3^0-o_2^0)\\
&=
\begin{pmatrix}
0 \\ 0 \\ 1
\end{pmatrix}\times
\begin{pmatrix}
(C1C2-S1S2)l_3 C3-(C1S2+S1C2)l_3 S3 \\ (S1C2+C1S2)l_3 C3-(S1S2-C1C2)l_3 S3 \\ 0
\end{pmatrix}\\
&=
\begin{pmatrix}
-((S1C2+C1S2)l_3 C3-(S1S2-C1C2)l_3 S3) \\ (C1C2-S1S2)l_3 C3-(C1S2+S1C2)l_3 S3 \\ 0
\end{pmatrix}\\
J_{v_3} &= 
z_2\times (o_3^0-o_3^0)
=
\begin{pmatrix}
0 \\ 0 \\ 0
\end{pmatrix}\\
J_{w_1} &= J_{w_2} = J_{w_3} = 
\begin{pmatrix}
0 \\ 0 \\ 1
\end{pmatrix}
\end{align*}

The Jacobian is then:

{\scriptsize
\begin{flalign*}
J &=
\begin{pmatrix}
(C1C2-S1S2)l_3 C3-(C1S2+S1C2)l_3 S3+C1l_2C2-S1l_2S2 & -((S1C2+C1S2)l_3 C3-(S1S2-C1C2)l_3 S3) & 0 \\
(S1C2+C1S2)l_3 C3-(S1S2-C1C2)l_3 S3+S1l_2C2+C1l_2S2 & (C1C2-S1S2)l_3 C3-(C1S2+S1C2)l_3 S3 &  0 \\
0 & 0 & 0 \\
0 & 0 & 0 \\
0 & 0 & 0 \\
1 & 1 & 1
\end{pmatrix}	
\end{flalign*}
}
